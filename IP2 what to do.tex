\documentclass[10pt,a4paper,draft]{article}
\usepackage[utf8]{inputenc}
\usepackage{amsmath}
\usepackage{amsfonts}
\usepackage{amssymb}
\author{Jim van Abkoude \\ Guus Beckett}
\title{IP2 - What to do}
\begin{document}
\maketitle
\newpage
\section{DokterClient}

\begin{description}
\item[Knop om test te starten] Button die een start signaal stuurt naar de cli\"enten client.
\item[Resultaat weergeven] Een popup form dat naar voren komt zodra de test is afgerond.
\item[Oude resultaten weergeven] Een form dat de gebruiker een gebruikersnaam en een datum laat invullen waarna de data die aan die eisen voldoet weer wordt gegeven in een form zoals beschreven in punt 2.
\end{description}
\newpage
\section{Cli\"enten client}
\begin{description}
\item[Uitvoeren van de test] De code voor de test komt in dit programma te staan, net als de instructies voor de assistent en cli\"ent.
\item[Instructies] De instructies voor de cli\"ent en de assistent worden weergegeven en er wordt duidelijk, per onderdeel, beschreven wat er van de cli\"ent en assistent verwacht wordt.
\end{description}
\newpage
\section{Server}
\begin{description}
\item[Testdata opslaan na test] Opslaan van data die getest is in een binary file.
\item[Versturen van oude data] Versturen van oude data op basis van de naam van de cli\"ent en een datum. 
\end{description}
\newpage
\end{document}